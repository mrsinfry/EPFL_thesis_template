\cleardoublepage
\chapter{Introduction}
\markboth{Introduction}{Introduction}
\addcontentsline{toc}{chapter}{Introduction}

\section{AlScN as the material of the future}
Since the beginning of the decade Aluminum-Scandium Nitride became more and more a common buzzword in scientific publications. The main reasons behind the increased interest on this novel materials are two: IC fabrication compatibility and most importantly increased piezoelectric coupling. 


\section{Material properties }
The first occurrency of an alloying of aluminum and scandium was in 1971 \cite{caro_piezoelectric_2015} and its first application was to improve the strain resistance of Al in aeronautics application. With the advent of miniaturisation in FR front-ends the bulky Quartz crystal started to be replace by MEMS-based resonator such as SAW, BAW, or CMR. The reason being that MEMS devices have a higher throughput and a lower footprint than quartz crystals, allowing for batch fabrication and integrability, still keeping the higher Q that mechanical systems show compared to LC tanks. In parallel to the architecture evolution of RF MEMS a range of new materials to replace quartz have been investigated, PZT, ZnO, AlN. The latter is the most important one to understand how the evolution of AlScN.
\subsection{AlN}
Aluminum Nitride is a binary nitride ceramic whose wirtzite phase exhibits piezoelectric properties. AlN can be deposited on a proper seed layer so that the growth is oriented in the c-axis (the perpendicular to surface axis) in a wurtzite phase. The crystal phase depends on the seed layer over which AlN is deposited. Literature \cite{xiong_influence_2010} shows that the preferential growth substrate material for AlN is Pt, due to the lattice amtching between the Hexagonal structure of AlN and the cubic phase of Pt. AlN  

\section{Deposition of AlScN}
\section{Types of MEMS Resonator}
