\cleardoublepage
\chapter{Introduction}
\markboth{Introduction}{Introduction}
\addcontentsline{toc}{chapter}{Introduction}

\section{AlScN as the material of the future}
Since the beginning of the decade Aluminum-Scandium Nitride became more and more a common buzzword in scientific publications. The main reasons behind the increased interest on this novel materials are two: IC fabrication compatibility and most importantly increased piezoelectric coupling. 


\section{Material properties }
The first occurrency of an alloying of aluminum and scandium was in 1971 \cite{caro_piezoelectric_2015} and its first application was to improve the strain resistance of Al in aeronautics application. With the advent of miniaturisation in FR front-ends the bulky Quartz crystal started to be replace by MEMS-based resonator such as SAW, BAW, or CMR. The reason being that MEMS devices have a higher throughput and a lower footprint than quartz crystals, allowing for batch fabrication and integrability, still keeping the higher Q that mechanical systems show compared to LC tanks. In parallel to the architecture evolution of RF MEMS a range of new materials to replace quartz have been investigated, PZT, ZnO, AlN. The latter is the most important one to understand how the evolution of AlScN.

\subsection{AlN}
Aluminum Nitride is a binary nitride ceramic whose wirtzite phase exhibits piezoelectric properties. AlN can be deposited on a proper seed layer so that the growth is oriented in the c-axis (the perpendicular to surface axis) in a wurtzite phase. The crystal phase depends on the seed layer over which AlN is deposited. Literature \cite{xiong_influence_2010} shows that the preferential growth substrate material for AlN is Pt, due to the lattice amtching between the Hexagonal structure of AlN and the cubic phase of Pt. 

\section{Deposition of AlScN}
Piezoelectricity in AlN is a consequence of the dipolar nature of the crystalline cell of wurtzite-type crystal. From the first studies using DFT \cite{caro_piezoelectric_2015} \cite{akiyama_enhancement_2009} show that the doping of aluminum with scandium increases the piezoelectric coupling coefficient of AlN. The reason lies in the lattice distortion induced by Sc in AlN, causing a structural phase transition. According to \cite{akiyama_enhancement_2009} a second effect on the piezoelectricity improvement lies in the hybridisation of ionic into covalent bond, due to the lower electronegativity of Sc compared to Al (1.36 vs 1.61). An increase of the concentration of Sc results in an enhancement of response up to 43\% Sc followed by a drastic performance drop due to the crystallisation in a rock-salt struture typical of Scandium rather than the wurtzite lattice of AlN. The actual state-of art includes depositions carried with Sputtering \cite{heinz_sputter_2017} \cite{colombo_investigation_2017} \cite{felmetsger_sputter_2019}, co-sputtering \cite{akiyama_enhancement_2009}, Molecular Beam Epitaxy \cite{park_epitaxial_nodate} \cite{casamento_physical_nodate} \cite{hardy_epitaxial_nodate}, Metal-Organic CVD \cite{leone_metal-organic_2020} 

\subsection{Sputtering of AlScN}
The first and most simple to describe method for AlScN deposition is sputtering from an alloyed target using reactive sputtering. In this case the Al-Sc percentages are decided during the target fabrication. This technology has been used to fabricate targets with 17.5\% Sc concentration \cite{lozzi_al083sc017n_2019}, 20\% Sc concentration \cite{colombo_investigation_2017}, a various range of concentration from 6.5\% to 28\% \cite{heinz_sputter_2017}, 40\% \cite{sandu_impact_2020}. The upper bound of 40\% for a sputtering target comes from the difficulty in alloying aluminum and scandium to form a uniform target, which is critical for a high quality deposition. The absence of higher concentration AlSc targets is nevertheless not critical because according to \cite{akiyama_influence_2009} at a concentration higher than 43\% scandium AlScN loses their benefits in term of piezoelectric response. Sputtering is carried out in a chamber with a variable concentration of reactive Nitrogen and inert Argon. Piezoelectric response in thin films depends on the percentage of nitrogen in the sputtering athmosphere \cite{akiyama_preparation_2010} as the Al and Sc ions react with N to deposit AlScN films.

\subsection{Co-Sputtering of AlScN}
To overcome the absence of targets, or more in general to achieve a larger flexibility in the Al-Sc ratio, Co-sputtering is a solution. Rather than using a single alloyed target a dual set-up machine with multiple targets allow to change the film composition by changing the sputtering power of each target. The delicate part of co-sputtering lies in surface uniformity as the sputtered particles are impacting the wafer from different angles. In literature a Sc percentage up to 46\% \cite{lu_development_2019} has been reached using co-sputtering.

\subsection{MBE of AlScN}
Being sputtering and co-sputtering the most widely used methods for AlScN depostion the majoritz of literature follows these two approaches. Nevertheless, in cases when it is necessary to achieve high cristallinity, a more advanced approach using molecular beam epitaxy has been sused \cite{hardy_epitaxial_2017}. In the quoted example, AlScN thin film was grown on a buffer GaN layer over SiC for a HEMT structure.

\section{Types of MEMS Resonator}
\label{sec:resonators}

The use of MEMS fabrication technologies allows for different architectures of resonators to be used, each with its adavantages. In general, every resonator has the following elements: two electrodes to transfer signal to and from the resonator, an active layer in piezoelectric material which converts the electric field into discplacement and vice versa, and a clamping to the substrate.

\subsection{SAW}
Surface acoustic wave resonators 