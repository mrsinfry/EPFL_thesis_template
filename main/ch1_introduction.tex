\cleardoublepage
\chapter{Introduction}
\markboth{Introduction}{Introduction}
\addcontentsline{toc}{chapter}{Introduction}

\section{AlScN as the material of the future}
Since the beginning of the decade Aluminum-Scandium Nitride became more and more a common buzzword in scientific publications. The main reasons behind the increased interest on this novel materials are two: IC fabrication compatibility and most importantly increased piezoelectric coupling. 


\section{Material properties }
The first occurrency of an alloying of aluminum and scandium was in 1971 \cite{caro_piezoelectric_2015} and its first application was to improve the strain resistance of Al in aeronautics application. With the advent of miniaturisation in FR front-ends the bulky Quartz crystal started to be replace by MEMS-based resonator such as SAW, BAW, or CMR. The reason being that MEMS devices have a higher throughput and a lower footprint than quartz crystals, allowing for batch fabrication and integrability, still keeping the higher Q that mechanical systems show compared to LC tanks. In parallel to the architecture evolution of RF MEMS a range of new materials to replace quartz have been investigated, PZT, ZnO, AlN. The latter is the most important one to understand how the evolution of AlScN.

\subsection{AlN}
Aluminum Nitride is a binary nitride ceramic whose wirtzite phase exhibits piezoelectric properties. AlN can be deposited on a proper seed layer so that the growth is oriented in the c-axis (the perpendicular to surface axis) in a wurtzite phase. The crystal phase depends on the seed layer over which AlN is deposited. Literature \cite{xiong_influence_2010} shows that the preferential growth substrate material for AlN is Pt, due to the lattice amtching between the Hexagonal structure of AlN and the cubic phase of Pt. 

\section{Deposition of AlScN}
Piezoelectricity in AlN is a consequence of the dipolar nature of the crystalline cell of wurtzite-type crystal. From the first studies using DFT \cite{caro_piezoelectric_2015} \cite{akiyama_enhancement_2009} show that the doping of aluminum with scandium increases the piezoelectric coupling coefficient of AlN. The reason lies in the lattice distortion induced by Sc in Aln, causing a structural phase transition. According to \cite{akiyama_enhancement_2009} a second effect on the piezoelectricity improvement lies in the hybridisation of ionic into covalent bond, due to the lower electronegativity of Sc compared to Al (1.36 vs 1.61). An increase of the concentration of Sc results in an enhancement of response up to 43\% Sc followed by a drastic performance drop. The actual state-of art includes depositions carried with Sputtering \cite{heinz_sputter_2017} \cite{colombo_investigation_2017} \cite{felmetsger_sputter_2019}, co-sputtering \cite{akiyama_enhancement_2009}, Molecular Beam Epitaxy \cite{park_epitaxial_nodate} \cite{casamento_physical_nodate} \cite{hardy_epitaxial_nodate}, Metal-Organic CVD \cite{leone_metal-organic_2020} 

\subsection{Sputtering of AlScN}
The first and most simple to describe method for AlScN deposition is sputtering from an alloyed target using reactive sputtering. 

\subsection{Co-Sputtering of AlScN}

\subsection{MBE of AlScN}

\subsection{MOCVD of AlScN}


\section{Types of MEMS Resonator}
